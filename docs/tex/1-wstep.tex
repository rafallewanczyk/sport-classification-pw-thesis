\newpage % Rozdziały zaczynamy od nowej strony.
\section{Wstęp}

\subsection{Problem}
Wraz z bardzo szybkim rozwojem technologii, smartfony oraz inne urządzenia przenośne stały się wszechobecne. Skutkiem tego jest stale rosnąca liczba danych wizualnych, takich jak zdjęcia oraz nagrania. Zgodnie z najnowszymi raportami firmy Sandvine\cite{sandvine}, aż 65\% całkowitego ruchu internetowego jest generowane właśnie przez treści wideo. Co więcej, przewiduje się, że ten odsetek będzie nadal rósł.

Semantyczny opis treści wideo umożliwia bardziej efektywne przeglądanie i wyszukiwanie ogromnych zbiorów, takich jak YouTube czy Instagram, dzięki lepszemu indeksowaniu i personalizacji treści. To pozwala użytkownikom odkrywać treści zgodne z ich zainteresowaniami, co zwiększa zaangażowanie i lojalność wobec platformy. Dzięki temu semantyczny opis przyczynia się do doskonalenia jakości usług i wzmocnienia konkurencyjności na rynku cyfrowym.

Klasyfikacja nagrań wideo może odbywać się na różnych poziomach. Na najwyższym poziomie można określić ogólną tematykę lub rodzaj przedstawionych treści, na przykład podział na kreskówki, sporty czy nagrania pochodzące z kamer monitoringu. Na kolejnym poziomie, działając w kontekście określonej klasy, można skupić się na konkretnych podklasach, np. dla nagrań sportowych dokonać klasyfikacji na konkretne dyscypliny, lub dla nagrań pochodzących z monitoringu podziału na sytuacje niebezpieczne oraz nie stanowiące zagrożenia. Istnieje również możliwość jeszcze bardziej szczegółowej klasyfikacji, gdzie dla każdego z nagrania określa się szczegółową akcję zawartą w jego fragmencie. W przypadku dyscyplin sportowych mogą to być konkretne akcje, z których składają się dane sporty, takie jak serwisy, odbiory, uderzenia forehandowe czy backhandowe w tenisie, a także ataki i bloki w siatkówce. 

\subsection{Zastosowania}
Tego typu systemy mają niezwykle szerokie zastosowanie zarówno w przetwarzaniu danych strumieniowych w czasie rzeczywistym, jak i w analizie wcześniej przygotowanych zbiorów. Przykładowe zastosowania systemów klasyfikacji akcji:
\begin{itemize}
    \item Etykietowanie zbiorów danych wideo, w celu ułatwienia wyszukiwania oraz serwowania treści dopasowanych do zainteresowań użytkownika
    \item Wyszukiwanie kluczowych elementów w rozgrywce sportowej, takich jak strzelenie bramki, skuteczny atak. Znajdowanie tego typu akcji pozwala na ich głębszą analizę oraz wytwarzanie skrótów meczów
    \item Aplikacje mające na celu poprawienia techniki wykonywanych sportów. W tym przypadku model może specjalizować się w rozpoznawaniu pojedynczych dyscyplin oraz rozpoznawać w jakim stopniu wykonywana czynność jest zbliżona do tej samej czynności wykonywanej przez specjalistę. 
    \item Rozpoznawanie na żywo niebezpiecznych sytuacji w nagraniach pochodzących z kamer monitoringu, oraz informowanie w takich przypadkach służb specjalnych 
\end{itemize}
\subsection{Cel}
Celem projektu jest wytworzenie systemu potrafiącego na podstawie nagrania wideo określić, jaką przedstawia ono dyscyplinę sportu. 
